\documentclass[11pt]{article}
\usepackage{amsmath,amssymb,amsthm}
\DeclareMathOperator*{\E}{\mathbb{E}}
\let\Pr\relax
\DeclareMathOperator*{\Pr}{\mathbb{P}}
\newcommand{\eps}{\varepsilon}
\newcommand{\inprod}[1]{\left\langle #1 \right\rangle}
\newcommand{\R}{\mathbb{R}}
\newcommand{\handout}[5]{
  \noindent
  \begin{center}
  \framebox{
    \vbox{
      \hbox to 5.78in { {\bf #5 } \hfill #2 }
      \vspace{4mm}
      \hbox to 5.78in { {\Large \hfill #1  \hfill} }
      \vspace{2mm}
      \hbox to 5.78in { {\em #3 \hfill #4} }
    }
  }
  \end{center}
  \vspace*{4mm}
}
\newcommand{\headline}[5]{\handout{#1}{#2}{#3}{#4}{#5}}
\topmargin 0pt
\advance \topmargin by -\headheight
\advance \topmargin by -\headsep
\textheight 8.9in
\oddsidemargin 0pt
\evensidemargin \oddsidemargin
\marginparwidth 0.5in
\textwidth 6.5in
\parindent 0in
\parskip 1.5ex

\begin{document}

\headline{Project Proposal: Musical Accompaniment}
         {November 2, 2017}
         {}
         {Amy Gu and Brian Yu}
         {Computer Science 182}

In our project, we will design a real-time musical accompaniment system that
will attempt to match a human soloist's playing in terms of its tempo,
pitch, and dynamics.

\section{Overview}

\section{Course Topics}

\section{Expected Behavior}

\section{Issues to Focus On}

\section{Papers}

\section{Distribution of Work}

\end{document}
